% Options for packages loaded elsewhere
\PassOptionsToPackage{unicode}{hyperref}
\PassOptionsToPackage{hyphens}{url}
\PassOptionsToPackage{dvipsnames,svgnames,x11names}{xcolor}
%

\documentclass[
  twocolumn]{biophys-new-mod}

\papertype{Article}


\usepackage{amsmath,amssymb}
\usepackage{lmodern}
\usepackage{iftex}
\ifPDFTeX
  \usepackage[T1]{fontenc}
  \usepackage[utf8]{inputenc}
  \usepackage{textcomp} % provide euro and other symbols
\else % if luatex or xetex
  \usepackage{unicode-math}
  \defaultfontfeatures{Scale=MatchLowercase}
  \defaultfontfeatures[\rmfamily]{Ligatures=TeX,Scale=1}
\fi
% Use upquote if available, for straight quotes in verbatim environments
\IfFileExists{upquote.sty}{\usepackage{upquote}}{}
\IfFileExists{microtype.sty}{% use microtype if available
  \usepackage[]{microtype}
  \UseMicrotypeSet[protrusion]{basicmath} % disable protrusion for tt fonts
}{}
\usepackage{xcolor}
\setlength{\emergencystretch}{3em} % prevent overfull lines
\setcounter{secnumdepth}{-\maxdimen} % remove section numbering
% Make \paragraph and \subparagraph free-standing
\ifx\paragraph\undefined\else
  \let\oldparagraph\paragraph
  \renewcommand{\paragraph}[1]{\oldparagraph{#1}\mbox{}}
\fi
\ifx\subparagraph\undefined\else
  \let\oldsubparagraph\subparagraph
  \renewcommand{\subparagraph}[1]{\oldsubparagraph{#1}\mbox{}}
\fi


\providecommand{\tightlist}{%
  \setlength{\itemsep}{0pt}\setlength{\parskip}{0pt}}\usepackage{longtable,booktabs,array}
\usepackage{calc} % for calculating minipage widths
% Correct order of tables after \paragraph or \subparagraph
\usepackage{etoolbox}
\makeatletter
\patchcmd\longtable{\par}{\if@noskipsec\mbox{}\fi\par}{}{}
\makeatother
% Allow footnotes in longtable head/foot
\IfFileExists{footnotehyper.sty}{\usepackage{footnotehyper}}{\usepackage{footnote}}
\makesavenoteenv{longtable}
\usepackage{graphicx}
\makeatletter
\def\maxwidth{\ifdim\Gin@nat@width>\linewidth\linewidth\else\Gin@nat@width\fi}
\def\maxheight{\ifdim\Gin@nat@height>\textheight\textheight\else\Gin@nat@height\fi}
\makeatother
% Scale images if necessary, so that they will not overflow the page
% margins by default, and it is still possible to overwrite the defaults
% using explicit options in \includegraphics[width, height, ...]{}
\setkeys{Gin}{width=\maxwidth,height=\maxheight,keepaspectratio}
% Set default figure placement to htbp
\makeatletter
\def\fps@figure{htbp}
\makeatother

\makeatletter
\let\oldlt\longtable
\let\endoldlt\endlongtable
\def\longtable{\@ifnextchar[\longtable@i \longtable@ii}
\def\longtable@i[#1]{\begin{figure}[t]
\onecolumn
\begin{minipage}{0.5\textwidth}
\oldlt[#1]
}
\def\longtable@ii{\begin{figure}[t]
\onecolumn
\begin{minipage}{0.5\textwidth}
\oldlt
}
\def\endlongtable{\endoldlt
\end{minipage}
\twocolumn
\end{figure}}
\makeatother
\makeatletter
\makeatother
\makeatletter
\makeatother
\makeatletter
\@ifpackageloaded{caption}{}{\usepackage{caption}}
\AtBeginDocument{%
\ifdefined\contentsname
  \renewcommand*\contentsname{Table of contents}
\else
  \newcommand\contentsname{Table of contents}
\fi
\ifdefined\listfigurename
  \renewcommand*\listfigurename{List of Figures}
\else
  \newcommand\listfigurename{List of Figures}
\fi
\ifdefined\listtablename
  \renewcommand*\listtablename{List of Tables}
\else
  \newcommand\listtablename{List of Tables}
\fi
\ifdefined\figurename
  \renewcommand*\figurename{Figure}
\else
  \newcommand\figurename{Figure}
\fi
\ifdefined\tablename
  \renewcommand*\tablename{Table}
\else
  \newcommand\tablename{Table}
\fi
}
\@ifpackageloaded{float}{}{\usepackage{float}}
\floatstyle{ruled}
\@ifundefined{c@chapter}{\newfloat{codelisting}{h}{lop}}{\newfloat{codelisting}{h}{lop}[chapter]}
\floatname{codelisting}{Listing}
\newcommand*\listoflistings{\listof{codelisting}{List of Listings}}
\makeatother
\makeatletter
\@ifpackageloaded{caption}{}{\usepackage{caption}}
\@ifpackageloaded{subcaption}{}{\usepackage{subcaption}}
\makeatother
\makeatletter
\@ifpackageloaded{tcolorbox}{}{\usepackage[many]{tcolorbox}}
\makeatother
\makeatletter
\@ifundefined{shadecolor}{\definecolor{shadecolor}{rgb}{.97, .97, .97}}
\makeatother
\makeatletter
\makeatother
\ifLuaTeX
  \usepackage{selnolig}  % disable illegal ligatures
\fi
\IfFileExists{bookmark.sty}{\usepackage{bookmark}}{\usepackage{hyperref}}
\IfFileExists{xurl.sty}{\usepackage{xurl}}{} % add URL line breaks if available
\urlstyle{same} % disable monospaced font for URLs
\hypersetup{
  pdftitle={What a title!},
  pdfauthor={Jane Doe; John Doe},
  pdfkeywords={Research, Cool Stuff},
  colorlinks=true,
  linkcolor={blue},
  filecolor={Maroon},
  citecolor={Blue},
  urlcolor={Blue},
  pdfcreator={LaTeX via pandoc}}


\title{What a title!}
\runningtitle{What a title!}

  \author[1,2,*]
  {Jane Doe}
  \author[1]
  {John Doe}

\affil[1]{Institute for Cool Things}
\affil[2]{University of Awesome Research}

  \corrauthor[*]{correspondign email}

\runningauthor{Jane Doe, John Doe}

\begin{document}
\begin{frontmatter}

\begin{abstract}
TODO Look at all the cool suff we found out!
\end{abstract}


\begin{sigstatement}
TODO This is very important, because\ldots{} Each manuscript must also
have a statement of significance or no more than 120 words.
\end{sigstatement}


\end{frontmatter}\ifdefined\Shaded\renewenvironment{Shaded}{\begin{tcolorbox}[frame hidden, interior hidden, borderline west={3pt}{0pt}{shadecolor}, breakable, sharp corners, boxrule=0pt, enhanced]}{\end{tcolorbox}}\fi

\hypertarget{introduction}{%
\section{Introduction}\label{introduction}}

This template is based on the Overleaf template provided by the
Biphysical Journal:
\url{https://www.overleaf.com/articles/biophysical\%E2\%80\%90journaltemplate/pxxcptphxdhv}
Your introduction goes here! Some examples of commonly used commands and
features are listed below, to help you get started. Leave a blank line
between blocks of text to start a new paragraph. Abbreviations should be
defined in the text at first mention.

Please also take note of the \texttt{\textbackslash{}section*\{...\}}
titles in this template: they are the required sections in a regular
research Article manuscript.

In particular, the main text of regular Articles and Computational Tools
manuscripts must be structured with the following sections:
\textbf{Introduction}, \textbf{Materials and Methods}, \textbf{Results},
\textbf{Discussion (or Results and Discussion)}, \textbf{Conclusion}.

Theoretical manuscripts may include just a \textbf{Methods} section and
do not require \textbf{Materials}.

No particular organization structure is required for Letters.

If your manuscript is accepted, the Biophysical production team will
re-format the references for publication. \emph{It is not necessary to
format the reference list yourself to mirror the final published form.}

\hypertarget{materials-and-methods}{%
\section{Materials and Methods}\label{materials-and-methods}}

Capitalize trade names and give manufacturers' full names and addresses
(city and state).

\hypertarget{sectioning-commands}{%
\subsection{Sectioning commands}\label{sectioning-commands}}

Lorem ipsum dolor sit amet, qui minim labore adipisicing minim sint
cillum sint consectetur cupidatat.

\hypertarget{figures-and-tables}{%
\subsection{Figures and Tables}\label{figures-and-tables}}

\hypertarget{tbl-numbers}{}
\begin{longtable}[]{@{}lr@{}}
\caption{\label{tbl-numbers}Look, numbers!}\tabularnewline
\toprule()
Thing & Value \\
\midrule()
\endfirsthead
\toprule()
Thing & Value \\
\midrule()
\endhead
A 42 & 18 \\
B 15 & 18 \\
C 34 & 17 \\
D 99 & 24 \\
\bottomrule()
\end{longtable}

\hypertarget{results}{%
\section{Results}\label{results}}

LaTeX is great at typesetting mathematics:

Let \(X_1, X_2, \ldots, X_n\) be a sequence of independent and
identically distributed random variables with \(\text{E}[X_i] = \mu\)
and \(\text{Var}[X_i] = \sigma^2 < \infty\), and let

\begin{equation}\protect\hypertarget{eq-clt}{}{
S_n = \frac{X_1 + X_2 + \cdots + X_n}{n}
      = \frac{1}{n}\sum_{i}^{n} X_i
}\label{eq-clt}\end{equation}

denote their mean. Then as \(n\) approaches infinity, the random
variables \(\sqrt{n}(S_n - \mu)\) converge in distribution to a normal
\(\mathcal{N}(0, \sigma^2)\). Thus concludes the explanation about
Equation~\ref{eq-clt}.

You can make lists with automatic numbering \ldots{}

\begin{enumerate}
\def\labelenumi{\arabic{enumi}.}
\item
  Like this,
\item
  and like this.
\end{enumerate}

\ldots or bullet points \ldots{}

\begin{itemize}
\item
  Like this,
\item
  and like this.
\end{itemize}

\ldots or with words and descriptions \ldots{}

\begin{description}
\item[Word]
Definition
\item[Concept]
Explanation
\item[Idea]
Text
\end{description}

An example quotation:

\begin{quote}
Lorem ipsum dolor sit amet, consectetur adipiscing elit, sed do eiusmod
tempor incididunt ut labore et dolore magna aliqua. Ut enim ad minim
veniam, quis nostrud exercitation ullamco laboris nisi ut aliquip ex ea
commodo consequat.
\end{quote}

\hypertarget{discussion}{%
\section{Discussion}\label{discussion}}

Lorem ipsum dolor sit amet, qui minim labore adipisicing minim sint
cillum sint consectetur cupidatat.

\hypertarget{conclusion}{%
\section{Conclusion}\label{conclusion}}

Sed ut perspiciatis unde omnis iste natus error sit voluptatem
accusantium doloremque laudantium, totam rem aperiam, eaque ipsa quae ab
illo inventore veritatis et quasi architecto beatae vitae dicta sunt
explicabo.

\hypertarget{author-contributions}{%
\section{Author Contributions}\label{author-contributions}}

Author1 designed the research. Author2 carried out all simulations,
analyzed the data. Author1 and Author2 wrote the article.

\hypertarget{acknowledgments}{%
\section{Acknowledgments}\label{acknowledgments}}

We thank G. Harrison, B. Harper, and J. Doe for their help.

\hypertarget{supplementary-material}{%
\section{Supplementary Material}\label{supplementary-material}}

An online supplement to this article can be found by visiting BJ Online
at \url{http://www.biophysj.org}.



\end{document}
